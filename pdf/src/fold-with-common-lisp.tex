\documentclass[dvipdfmx, 12pt]{beamer}
\usepackage{graphicx}
\usepackage{bxdpx-beamer}
\usepackage{pxjahyper}
\usepackage{minijs}
\usepackage{amsmath,amsfonts,amsthm,amssymb,amscd, euler}
\usepackage[all]{xy}
\renewcommand{\kanjifamilydefault}{\gtdefault}
\def\objectstyle{\displaystyle}
%-----------------------------------------------------------
\usetheme{CambridgeUS}
\usecolortheme{beaver}
%-----------------------------------------------------------
\title{重畳関数について with Common Lisp}
\author{MIYAO Satoaki}
\institute{}
\date{\today}

\begin{document}

\frame{\titlepage}

\section{目次}

\frame{
  \frametitle{目次}
  \begin{itemize}
    \item 自己紹介
    \item 重畳関数について
    \item 数学的な話
  \end{itemize}
}

\frame{
  \frametitle{自己紹介}
  \begin{itemize}
    \item Miyao Satoaki
    \item twitter: @myao\_s\_moking
    \item github: myaosato
    \item 最近、数学の方のイベントにも関わりだした
  \end{itemize}
}

\section{重畳関数について}

\frame{
  \frametitle{重畳関数}
  \begin{itemize}
    \item reduce お好きですか?
  \end{itemize}
}

\frame{
  \frametitle{reduce (hold)}
  \begin{itemize}
    \item 日本語では重畳関数
    \item reduceとか、holdと呼ばれます
    \item リスト(配列)の要素を順番に処理をして新しい値に変える関数
  \end{itemize}
}

\frame{
  \frametitle{検索キーワードとして}
  \begin{itemize}
    \item 検索すると関数型云々とか引っかかる
    \item 再帰呼び出しとも関係している
  \end{itemize}
}

\frame{
  \frametitle{reduceとは}
  \begin{itemize}
    \item (リストの)reduceの一般化
    \item 結局、重畳関数ってなによ
  \end{itemize}
}

\frame{
  \frametitle{reduce(リスト版)}
  \begin{itemize}
    \item 要素を順番に処理
    \item オプションや別名の関数で、処理する方向が逆に
    \item デモ
  \end{itemize}
}

\frame{
  \frametitle{reduce(リスト版)}
  \begin{itemize}
    \item リストを作り変えるという操作
    \item コンストラクタと強い関係がある(holdr)
  \end{itemize}
}

\frame{
  \frametitle{reduce(tree版)}
  \begin{itemize}
    \item 同じように木構造にも適用できる
    \item デモ
  \end{itemize}
}

\frame{
  \frametitle{reduce(自然数版)}
  \begin{itemize}
    \item 自然数なんかにも使える
    \item 自然数とは、0から始まって1, 2, 3と続く数
    \item デモ
  \end{itemize}
}

\section{数学的な話}

\frame{
  \frametitle{圏論って}
  \begin{itemize}
    \item 数学のある分野
    \item プログラミングとも関連している(らしい)
  \end{itemize}
}

\frame{
  \frametitle{圏}
  \[
    \xymatrix{
        A \ar@(u, l)_{1_A} \ar[dd]_{f} \ar[ddrr]^{g \circ f} \\
        & \\
        B \ar@(d, l)^{1_B} \ar[rr]_{g} & & C \ar@(d, r)_{1_C} \\
    }
  \]
  \begin{itemize}
    \item 矢印が射、 矢印の前後のアルファベットが対象を表す。
    \item 矢印は、合成が可能
  \end{itemize}
}

\frame{
  \frametitle{関手}
  \[
    \xymatrix{
        A \ar[dd]_{f} & & F(A) \ar[dd]^{F(f)} \\
        & \\
        B & & F(B) \\
    }
  \]
  \begin{itemize}
    \item 対象を対象に、射を射に移す対応
  \end{itemize}
}

\frame{
  \frametitle{F始代数}
  \[
    X \simeq F(X)
  \]
  \[
    \xymatrix{
        X \ar[ddd]_{f(h)} \ar@<-0.7ex>[rrr]_{in^{-1}} & & & F(X) \ar@<-0.7ex>[lll]_{in} \ar[ddd]^{F(f(h))} \\
        & & & \\
        & & & \\
        Y & & & F(Y) \ar[lll]_{h}
    }
  \]
  $f$がfoldに対応している.
}

\frame{
  \frametitle{各射について}
  \[
    \xymatrix{
        X \ar@<-0.7ex>[rrr]_{in^{-1}} & & & F(X) \ar@<-0.7ex>[lll]_{in}
    }
  \]
  \begin{itemize}
    \item $in$ が、コンストラクタ
    \item $in^{-1}$ がコンストラクタの逆
  \end{itemize}
}

\frame{
  \frametitle{各射について}
  \[
    \xymatrix{
        Y & & & F(Y) \ar[lll]_{h}
    }
  \]
  \begin{itemize}
    \item $h$ が、foldに渡す関数
  \end{itemize}
}

\frame{
  \frametitle{各射について}
  \[
    \xymatrix{
        F(X) \ar[ddd]_{F(f)} \\
        \\
        \\
        F(Y)
    }
  \]
  \begin{itemize}
    \item $F$は、関数を関数に移す。
    \item この時のルールをわかっていれば、foldをすぐに書けるようになる。
  \end{itemize}
}

\frame{
  \frametitle{多項式関手F(対象)}
  \begin{eqnarray*}
    F(X) &=& 1 + X \\
    F(X) &=& 1 + A \times X \\
    F(X) &=& B + A \times X \\
    F(X) &=& 1 + X \times X \\
    F(X) &=& A + X \times X \\
    F(X) &=& A + A \times X \times X
  \end{eqnarray*}
  \begin{itemize}
    \item $+$, $\times$は、足し算掛け算に似たようなもの
    \item その意味で、右辺が多項式になっているもの
  \end{itemize}
}

\frame{
  \frametitle{多項式関手F(射)}
  \begin{eqnarray*}
    F(X) &=& Id_1 + f \\
    F(X) &=& Id_1 + Id_A \times f \\
    F(X) &=& Id_B + Id_A \times f \\
    F(X) &=& Id_1 + f \times f \\
    F(X) &=& Id_A + f \times f \\
    F(X) &=& Id_A + Id_A \times f \times f
  \end{eqnarray*}
  \begin{itemize}
    \item 前ページで$X$に対応していた場所は対応する関数に変わる
    \item ここで$+$は場合分け、$\times$は、それぞれに作用させるくらいの気持ち
  \end{itemize}
}


\frame{
  \frametitle{各射について}
  \[
    \xymatrix{
        X \ar[ddd]_{f(h)} \\
        \\
        \\
        Y
    }
  \]
  \begin{itemize}
    \item $f$ が、fold
  \end{itemize}
}

\frame{
  \frametitle{Listの場合}
  \[
    \xymatrix{
        X \ar[ddd]_{f} \ar@<-0.7ex>[rrr]_{in^{-1}} & & & 1 + A \times X \ar@<-0.7ex>[lll]_{in} \ar[ddd]^{F(f)} \\
        & & & \\
        & & & \\
        Y & & & 1 + A \times Y \ar[lll]_{h}
    }
  \]
}

\frame{
  \frametitle{Treeの場合}
  \[
    \xymatrix{
        X \ar[ddd]_{f} \ar@<-0.7ex>[rrr]_{in^{-1}} & & & A + X \times X \ar@<-0.7ex>[lll]_{in} \ar[ddd]^{F(f)} \\
        & & & \\
        & & & \\
        Y & & & A + Y \times Y \ar[lll]_{h}
    }
  \]
}

\frame{
  \frametitle{Natural Numberの場合}
  \[
    \xymatrix{
        X \ar[ddd]_{f} \ar@<-0.7ex>[rrr]_{in^{-1}} & & & 1 + X \ar@<-0.7ex>[lll]_{in} \ar[ddd]^{F(f)} \\
        & & & \\
        & & & \\
        Y & & & 1 + Y \ar[lll]_{h}
    }
  \]
}

\end{document}
